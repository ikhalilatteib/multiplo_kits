\centerline{\bf �ZET}
\addcontentsline{toc}{section}{�ZET}
\begin{center}
\textbf{Projenin Amac�}
\end{center} 
 \mbox{Bu projenin amac� Multiplo robot kullanarak uygulamalar ge�ekle�tirmesi}\\
 

\begin{center}
\textbf{Projenin Kapsam�}
\end{center}
 \mbox{Bu projenin kapsam�nda uzaktan deneteleyebilen ve mesafe �l�en bir robot ger�ekle�tirmesi }\\

\begin{center}
\textbf{Sonu�lar}
\end{center}
 \mbox{}\\
%%%%%%%%%%%%%%%%%%%%%%%%%%%%%%%%%%%%%%%ABSTRACT%%%%%%%%%%%%%%%%%%%%%%%%%%%%%%%%%%%%%%%%%%
\newpage
\centerline{\bf ABSTRACT}
\addcontentsline{toc}{section}{ABSTRACT}
\begin{center}
\textbf{Project Objective}
\end{center}

 \mbox{The objective of this project is to implement applications using Multiplo robot.}\\
 
%Bilecik �eyh Edebali University Department of Computer Engineering students is studied. The aim of project is work to create a template for writing \LaTeX\ writing the final report template in the Project, to be written.

\begin{center}
\textbf{Scope of Project}
\end{center}
 \mbox{Within the scope of this project, a robot that can remotely control and measure distance has  }\\
 \mbox{been realized.}
%Bilecik �eyh Edebali University Computer Engineering Department Bilecik need to create a project template Latex codes assignment, involves the use of. The first part of the project consists of two parts, \LaTeX's development have been studied and why it is preferred that the use of information provided in the information. Faculty of Engineering of the university for the second part, Sheikh Edebali Bilecik Project Paper \LaTeX\ codes and are included.

\begin{center}
\textbf{Results}
\end{center}
 \mbox{}\dotfill\\
  \mbox{}\dotfill
%As a result, Bilecik Sheikh Edebali University Computer Engineering
%Department prepared a document that students can build project reports into \LaTeX. 